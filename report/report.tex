\documentclass{article}
\usepackage{graphicx}
\usepackage{hyperref}


\hypersetup{
  colorlinks, linkcolor=red
}

\begin{document}


\title{cs838 Lab 2}
\author{Sek Cheong}

\maketitle

%\begin{abstract}
%The abstract text goes here.
%\end{abstract}

\section{Introduction}
The purpose of this experiment was to design a neural network with single hidden layer to predict protein secondary structures. The data set we used in this experiment was the \href{https://archive.ics.uci.edu/ml/datasets/Molecular+Biology+(Protein+Secondary+Structure)}{Molecular Biology (Protein Secondary Structure) Data Set} obtained from the UC Irvine Machine Learning Repository. We used convolution with window size 17 for the input of the neural network and train the network with the middle secondary structure. The overall accuracy of our network was about $62\%$ with on average less than 3 epochs. 

\section{The Data}
The data set from UC Irvine was composed of two files one for training and one for testing. We combine these two files into a single file. The combined file was then split into training set, tuning set, and test set.  The test set was collection of every 5th example in combined file. The tune set was a collection of every 6th example in the combined file. The rest of the examples in the combined file formed the training set. 

\section{The Experiment}
We used all the examples in the training set for each epoch of the training. The early stop criteria was set on the difference of accuracy on the tuning set of previous epoch and the current epoch. When the difference is greater or equal than $0.001$ we stop the training. We tested the network with various number of hidden units and learning rate. We also tested the network unregularized and regularized with learning rate, momentum, and weight decay. Following were out findings. 

\subsection{Unregularized}
this is sub section
%\begin{equation}
%    \label{simple_equation}
%    \alpha = \sqrt{ \beta }
%\end{equation}

%\subsection{Subsection Heading Here}
%Write your subsection text here.
\begin{center}
 \includegraphics[width=3.0in]{learningrate}
\end{center}

%\begin{figure}
%    \centering
%    \includegraphics[width=2.0in]{learningrate}
%    \caption{Simulation Results}
%    \label{simulationfigure}
%\end{figure}

\section{Conclusion}
Write your conclusion here.

\end{document}